%\subsection[rnd]{Random}

\begin{frame}[fragile]
  \frametitlecpp[11]{Pseudo-random number generators (PRNG)}
  \begin{block}{Concept}
    \begin{itemize}
      \item For generating pseudo-random numbers the STL offers:
      \begin{itemize}
        \item engines and engine adaptors to generate pseudo random bits
        \item distributions to shape these into numbers
        \item access to (potential) hardware entropy
      \end{itemize}
      \item All found in header \cppinline{<random>}
    \end{itemize}
  \end{block}
  \begin{exampleblock}{Example}
    \begin{cppcode}
      #include <random>
      std::random_device rd;
      std::default_random_engine engine{rd()};
      std::normal_distribution dist(5.0, 1.5); // µ, σ
      double r = dist(engine);
    \end{cppcode}
  \end{exampleblock}
\end{frame}

\begin{frame}[fragile]
  \frametitlecpp[11]{Engines}
  \begin{block}{Engines}
    \begin{itemize}
      \item Generate random bits (as integers), depending on algorithm
      \item Can be seeded via constructor or \cppinline{seed()}
      \item Algorithms:
        {\footnotesize \cppinline{linear_congruential_engine}, \cppinline{mersenne_twister_engine}, \cppinline{subtract_with_carry_engine}}
      \item Adaptors:
        {\footnotesize \cppinline{discard_block_engine}, \cppinline{independent_bits_engine}, \cppinline{shuffle_order_engine}}
      \item Aliases:
        {\footnotesize \cppinline{minstd_rand0}, \cppinline{minstd_rand}, \cppinline{mt19937}, \cppinline{mt19937_64}, \cppinline{ranlux24}, \cppinline{ranlux48}, \cppinline{knuth_b}} (use these)
      \item \cppinline{std::default_random_engine} aliases one of the above
    \end{itemize}
  \end{block}
  \begin{exampleblock}{Engine creation with seed}
    \begin{cppcode}
      std::default_random_engine engine{42}; // or empty
      auto i = engine(); // operator(): random integer
    \end{cppcode}
  \end{exampleblock}
\end{frame}

\begin{frame}[fragile]
  \frametitlecpp[11]{Determinism and random device}
  \begin{block}{Engines / PRNGs}
    \begin{itemize}
      \item Are \emph{deterministic}
      \item Will produce the same number sequence for the same seed
      \item May be useful for reproducibility
    \end{itemize}
  \end{block}
  \begin{block}{\texttt{std::random\_device}}
    \begin{itemize}
      \item Provides \emph{non-deterministic} random numbers
      \item Uses hardware provided entropy, if available
      \item May become slow when called too often,\\e.g.\ when hardware entropy pool is exhausted
      \item Use only to seed engine, if \emph{non-deterministic} numbers needed
    \end{itemize}
  \end{block}
  \begin{exampleblock}{Non-deterministic engine seeding}
    \begin{cppcode}
      std::random_device rd;
      std::default_random_engine engine{rd()};
    \end{cppcode}
  \end{exampleblock}
\end{frame}

\begin{frame}[fragile]
  \frametitlecpp[11]{Distributions}
  \begin{block}{Distributions}
    \begin{itemize}
      \item Take random bits from engines and shape them into numbers
      \item Standard library provides a lot of them
      \begin{itemize}
        \item Uniform, bernoulli, poisson, normal and sampling distributions
        \item See \href{https://en.cppreference.com/w/cpp/numeric/random}{cppreference} for the full list
      \end{itemize}
    \end{itemize}
  \end{block}
  \begin{exampleblock}{Distributions sharing a non-deterministic engine}
    \begin{cppcode*}{gobble=2}
      std::random_device rd;
      std::default_random_engine engine{rd()};
      std::uniform_int_distribution<int> dist(2,7);//min,max
      const int size = dist(engine); // gen. random number
      std::vector<int> v(size);
      std::normal_distribution<double> norm(5.0, 1.5); //µ,σ
      std::generate(begin(v), end(v),
                    [&]{ return norm(engine); });
    \end{cppcode*}
  \end{exampleblock}
\end{frame}

\begin{frame}[fragile]
  \frametitlecpp[98]{C random library}
  \begin{block}{C random library}
    \begin{itemize}
      \item The C library (\cppinline{<cstdlib>}) offers a primitive PRNG:
      \item \cppinline{rand()} returns random int between 0 and \cppinline{RAND_MAX} (incl.)
      \item \cppinline{srand(seed)} sets the global seed
    \end{itemize}
  \end{block}
  \begin{alertblock}{Disadvantages}
    \begin{itemize}
      \item Seeding is not thread-safe; \cppinline{rand()} may be.
      \item Returned numbers have small, implementation-defined range
      \item No guarantee on quality
      \item Mapping a random number to a custom range is hard
      \item E.g.: \cppinline{rand() % 10} yields biased numbers and is wrong!
    \end{itemize}
  \end{alertblock}
  \begin{goodpracticeWithShortcut}{Strongly avoid the C random library}{C random library}
    Use the \cpp11 facilities from the \cppinline{<random>} header
  \end{goodpracticeWithShortcut}
\end{frame}
