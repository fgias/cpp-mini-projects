\subsection[ADL]{Name Lookups}

\begin{frame}[fragile]
  \frametitlecpp[98]{Basics of name lookup}
  \begin{exampleblock}{Example code}
    \begin{cppcode}
      std::cout << std::endl;
    \end{cppcode}
  \end{exampleblock}
  \begin{block}{How to find the declaration of a name?}
    Mainly 2 cases :
    \begin{itemize}
    \item qualified lookup, for names preceded by `::'
      \begin{itemize}
      \item here \cppinline{cout} and \cppinline{endl}
      \item name is only looked for in given class/namespace/enum class
      \end{itemize}
    \item unqualified lookup
      \begin{itemize}
      \item here for \cppinline{std} and \cppinline{operator<<}
      \item name is looked for in a sequence of scopes until found
      \item remaining scopes are not examined
      \end{itemize}
    \end{itemize}
  \end{block}
\end{frame}

\begin{frame}
  \frametitlecpp[98]{Unqualified name lookup and ADL}
  \begin{block}{Ordered list of scopes (simplified)}
    \begin{itemize}
    \item file (only for global level usage)
    \item current namespace/block, enclosing namespaces/blocks, etc...
    \item current class if any, base classes if any, etc...
    \item for a call expression (e.g.\ \cppinline{f(a, b)} or \cppinline{a + b}), Argument Dependent Lookup (ADL)
    \end{itemize}
  \end{block}
  \begin{exampleblock}{Argument Dependent Lookup (simplified)}
    To find a function name (including operators), the compiler also examines the arguments. For each argument, it searches:
    \begin{itemize}
    \item class, if any
    \item direct and indirect base classes, if any
    \item enclosing namespace
    \end{itemize}
  \end{exampleblock}
\end{frame}

\begin{frame}[fragile]
  \frametitlecpp[98]{ADL consequences (1)}
  \begin{block}{Use standalone/non-member functions}
    When a method is not accessing the private part of a class, make it a function in the same namespace
    \vspace{-1mm}
    \begin{columns}[T]
      \begin{column}{.35\textwidth}
        Don't write :
        \vspace{-1mm}
        \begin{cppcode*}{gobble=6}
          namespace MyNS {
            struct A {
              T func(...);
            };
          }
        \end{cppcode*}
      \end{column}
      \begin{column}{.35\textwidth}
        Prefer :
        \vspace{-1mm}
        \begin{cppcode*}{gobble=6,firstnumber=6}
          namespace MyNS {
            struct A { ... };
            T func(const A&, ..);
          }
        \end{cppcode*}
      \end{column}
    \end{columns}
    \vspace{.2cm}
    Advantages :
    \begin{itemize}
    \item minimal change in user code, \cppinline{func} still feels part of class \cppinline{A}
    \item makes sure \cppinline{func} does not touch internal state of \cppinline{A}
    \end{itemize}
    Notes :
    \begin{itemize}
    \item non-member \cppinline{func} has to be in same namespace as \cppinline{A}
    \item please avoid global namespace
    \end{itemize}
  \end{block}
\end{frame}

\begin{frame}[fragile]
  \frametitlecpp[98]{ADL consequences (2)}
  \begin{block}{Customization points and \texttt{using}}
    \begin{columns}[t]
      \begin{column}{.35\textwidth}
        Don't write :
        \begin{cppcode*}{gobble=6}
          N::A a,b;
          std::swap(a, b);
        \end{cppcode*}
      \end{column}
      \begin{column}{.35\textwidth}
        Prefer :
        \begin{cppcode*}{gobble=6,firstnumber=3}
          N::A a,b;
          using std::swap;
          swap(a, b);
        \end{cppcode*}
      \end{column}
    \end{columns}
    \vspace{.2cm}
    Advantages :
    \begin{itemize}
    \item allows to use \cppinline{std::swap} by default
    \item but benefits from any dedicated overload
    \end{itemize}
    \begin{columns}
      \begin{column}{.7\textwidth}
        \begin{cppcode*}{gobble = 6,firstnumber=6}
          namespace N {
            class A { ... };
            // optimized swap for A
            void swap(A&, A&);
          }
        \end{cppcode*}
      \end{column}
    \end{columns}
  \end{block}
\end{frame}

\begin{frame}[fragile]
  \frametitlecpp[11]{Hidden Friends}
  \begin{block}{Hidden friends}
    \begin{itemize}
      \item Friend functions \emph{defined} in class scope are ``hidden friends''
      \begin{itemize}
        \item Without any declaration outside the class
      \end{itemize}
      \item They are \emph{not} member functions (no access to \cppinline{this})
    \end{itemize}
  \end{block}
  \begin{exampleblock}{\texttt{operator<<} as hidden friend}
    \small
    \begin{cppcode*}{gobble=2}
      class Complex {
        float m_real, m_imag;
      public:
        Complex(float, float) { ... }
        friend
        std::ostream & operator<<(std::ostream & os,
                                  Complex const & c) {
          return os << c.m_real << " + " << c.m_imag << "i";
        }
      };
      std::cout << Complex{2.f, 3.f}; // Prints 2 + 3i
    \end{cppcode*}
  \end{exampleblock}
\end{frame}

\begin{frame}[fragile]
  \frametitlecpp[11]{Hidden Friends and ADL}
  \begin{block}{Advantages of hidden friends}
    \begin{itemize}
      \item Are \emph{only} found via ADL, e.g.\ \cppinline{std::cout << complex;}
      \item Compiler needs to consider less functions during ordinary name lookup (faster compilation)
      \item Avoid accidental implicit conversions
    \end{itemize}
  \end{block}
  \begin{alertblock}{Accidental conversions - two examples in one}
    \footnotesize
    \begin{cppcode*}{gobble=2}
      std::ostream & operator<< // out-of-class definition
        (std::ostream & os, Complex const & c) { ... }
      struct Fraction {
        int num, denom;
        operator Complex() const { ... } // case 1: conversion op
      };
      Complex::Complex(Fraction f); // or case 2: converting ctor
      std::cout << Fraction{2,4}; // Prints 0.5 + 0i, would call:
      //case 1: operator<<(std::cout, Fraction{2,4}.operator Complex());
      //case 2: operator<<(std::cout, Complex(Fraction{2, 3}));
    \end{cppcode*}
  \end{alertblock}
\end{frame}

\begin{frame}[fragile]
  \frametitlecpp[98]{Good practice for operator overloading}
  \begin{goodpractice}{Operator overloading}
    \begin{itemize}
      \item Must declare in-class, as member operators:
      \begin{itemize}
        \item \cppinline{operator=}, \cppinline{operator()}, \cppinline{operator[]}, \cppinline{operator->}
      \end{itemize}
      \item Prefer in-class declaration, as member operators:
      \begin{itemize}
        \item compound assignment operators, e.g.\ \cppinline{operator+=}
        \item unary operators
        \begin{itemize}
          \item e.g.\ \cppinline{operator++}, \cppinline{operator--}, \cppinline{operator!}, unary \cppinline{operator-} (negate), dereference operator \cppinline{operator*}
        \end{itemize}
      \end{itemize}
      \item Prefer definition in-class as hidden friends:
      \begin{itemize}
        \item binary arithmetic operators, e.g.\ \cppinline{operator+}, \cppinline{operator|}
        \item binary logical operators, e.g.\ \cppinline{operator<}
        \item stream operators, e.g.\ \cppinline{operator<<}
      \end{itemize}
      \item Avoid overloading:
      \begin{itemize}
        \item Address-of \cppinline{operator&}
        \item Boolean logic operators, e.g.\ \cppinline{operator||}
        \item Comma operator \cppinline{operator,}
        \item Member dereference operators \cppinline{operator->*}
      \end{itemize}
    \end{itemize}
  \end{goodpractice}
\end{frame}
